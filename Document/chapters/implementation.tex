%% TODO: Add a small paragraph to tell what this chapter is about
This chapter delves into the implementation of each module inside Test Genie system. Overall, this system consist of three main modules: 
\begin{itemize}
    \item[-] \textbf{Project Manager}: This module manages all the projects that are cloned to server. It mostly responsible for file-based activities and running CLI for each project.
    \item[-] \textbf{Business Logic Analyzer}: This module will take various source file from Project Manager and break the source code into smaller pieces (blocks). Then, it will analyze each block and determine what each block does and how it should be tested if possible. A test plan will also be generated for each block and save it to the database.
    \item[-] \textbf{Test Generator}: This module will take the test plan from Business Logic Analyzer and generate a test case for each block. The generated test cases will be saved as files directly in the project source code on server and can be used to run the tests later (validation).
\end{itemize}
Additionally, this system also have \textbf{DBMS} module to control the database but this module will not be explained thoroughly in this chapter.

%% TODO: Make sure to use \textbf or \textit for highlighting keywords, and \cite{} to cite the corresponding quotations

\section{Project Manager module}

The \textbf{ProjectManager} module serves as the core backend functionality for handling projects within the Test Genie system. It provides a robust framework for managing software projects by integrating Git-based repositories, file management, and testing workflows. The module is built around the \textit{Project} class, which encapsulates essential functionalities such as cloning repositories, recognizing project frameworks, and managing project files. Additionally, it features an abstract interface for test creation, validation, and execution, allowing for framework-specific extensions of functionality. For instance, the \textit{Flutter} subclass extends the \textit{Project} class to handle Flutter-specific tasks, including dependency management, `pubspec.yaml` parsing, and test execution. By modularizing these functionalities, the \textbf{ProjectManager} module streamlines project handling and enhances the system's scalability for various software development frameworks.

\subsection{Module prequisites}
This module require the SDK of supported frameworks to be installed standablone in folder \textit{./SDKs} inside the module folder. This design not only allows the module to be easily extended and modifiled to support other frameworks, but also avoid more SDK installation on the server OS. Since the \textit{Project} class (Listing A.1) just mainly control git management and file management, the subclass can freely control how the SDKs are used. \\

Subclass of Project are required to implement the following methods:
    \begin{itemize}
        \item[-] \textbf{create\_test}: This method will create the test file in the location that is required by the framework.
        \item[-] \textbf{get\_test\_content}: This method will return the content of the test file that is created by the \textbf{create\_test} method. The content of the test file is generated by the Business Logic Analyzer module.
        \item[-] \textbf{run\_test}: This method will run the test file that is created by the \textbf{create\_test} method. The test result will be returned to the caller.
        \item[-] \textbf{validate}: This method will run all the test files in the test directory and return the result. This method is used to validate the test files that are created by the \textbf{create\_test} method.
        \item[-] \textbf{getListSourceFiles}: This is an important method, which will partly decide how the source code is split into blocks. The starting point file (main file) should be placed on the first position of the list. The list will be used to split the source code into blocks. The list should contain all the source files in the project (relative to the project directory).
    \end{itemize}


\subsection{Flutter class}

The \textbf{Flutter} class extends the \textbf{Project} class to provide framework-specific support for managing Flutter projects. This class is responsible for handling operations unique to Flutter, such as managing dependencies, running tests, and validating projects. It ensures that the Flutter SDK is installed and properly configured in the \textit{./SDKs/flutter} directory before performing any operations.

Key methods of the \textbf{Flutter} class include:
\begin{itemize}
    \item[-] \textbf{\_runFlutterCLI}: This method executes commands using the Flutter CLI within the context of the project directory. It supports arguments for various Flutter commands and handles errors if the command fails.
    \item[-] \textbf{\_checkSDK}: Ensures that the Flutter SDK is installed and operational by running the \texttt{flutter --version} command. If the SDK is not present or misconfigured, the method raises an exception.
    \item[-] \textbf{\_flutterPubGet}: Automatically installs dependencies listed in the \textit{pubspec.yaml} file by running \texttt{flutter pub get}.
    \item[-] \textbf{\_addTestDependency}: Adds the Flutter \textit{test} package as a dependency using \texttt{flutter pub add test}.
    \item[-] \textbf{create\_test}: Creates a test file in the designated \textit{test} directory of the project. If the file already exists and overwriting is not allowed, an exception is raised.
    \item[-] \textbf{get\_test\_content}: Retrieves the content of a test file from the \textit{test} directory.
    \item[-] \textbf{run\_test}: Executes a specified Dart test file using the Flutter CLI and returns the results.
    \item[-] \textbf{validate}: Iterates through all Dart test files in the \textit{test} directory and validates them by running each test.
    \item[-] \textbf{getListSourceFiles}: Collects and returns a list of all source files in the \textit{lib} directory, ensuring that the \textit{main.dart} file is prioritized as the entry point.
\end{itemize}

This design enables seamless integration of Flutter-specific features into the \textbf{Test Genie} system while adhering to the modular structure defined by the \textbf{Project} class. By implementing these methods, the \textbf{Flutter} class ensures compatibility with the broader system and provides developers with a streamlined process for managing and testing Flutter projects.

\section{Business Logic Analyzer module}
The \textbf{Business Logic Analyzer} module is a critical component of the Test Genie system, designed to parse and analyze the source code of a project. It constructs a \textbf{Dependency Diagram} that represents the logical structure and relationships within the project. By leveraging framework-specific analysis strategies, such as the \textit{FlutterAnalyzeStrategy}, this module identifies functional blocks and their interconnections. Each block is further enriched with predictions generated by the \textbf{AI Agent}, which analyzes the code to provide insights into its behavior and logic. This modular design allows the Business Logic Analyzer to be easily extended to support additional frameworks, making it versatile and scalable for various software projects. The output of this module serves as the foundation for the subsequent test generation process.
\section{Test Generator module}

\section{Other implementations}