%% TODO: Make sure to use \textbf or \textit for highlighting keywords, and \cite{} to cite the corresponding quotations
This chapter presents a comprehensive evaluation of the \textbf{Test Genie} system, analyzing its performance characteristics, accuracy in generating test files, and comparing it with existing approaches. The evaluation aims to assess whether the system meets the requirements established in Chapter 3 and to identify both strengths and limitations of the implemented solution. By examining metrics related to execution time, algorithmic complexity, and test generation accuracy, this chapter provides insights into the practical viability of using AI-driven techniques for automated test generation in Flutter projects.

\section{Performance Analysis}

The performance of the \textbf{Test Genie} system is evaluated based on two most complex process: AI generation time and BLA algorithm complexity. Others processes are negletable since their time complexity are O(1). Although DBMS module is dependent on the server's response time, it is impossible to estimate the time complexity of the server's response. However, it is important to note that the DBMS module may become a bottleneck in the overall system performance.

To fully calculate the performance estimation of this system, let the estimation time for AI to generate tests is $T_{AI\_test}$ and the time to fully generate blocks in BLA is $T_{BLA}$. Since the block generating procedure contain two separate steps are source code splitting and blocks's prediction generation, we can define the time complexity of BLA as $T_{BLA} = T_{split} + T_{predict}$, where $T_{split}$ is the time complexity of source code splitting and $T_{predict}$ is the time complexity of blocks's prediction generation. The overall time complexity of the system can be expressed as:
\begin{equation}
T_{total} = T_{AI\_test} + T_{BLA} = T_{AI\_test} + T_{split} + T_{predict}
\end{equation}

If we account for DBMS module as a parameter $m$, the overall time complexity of the system can be expressed as:
\begin{equation}
T_{total} = T_{AI\_test} + T_{BLA} + m = T_{AI\_test} + T_{split} + T_{predict} + m
\end{equation}


\subsection{Code Splitting Algorithm complexity}

\subsection{AI generation time estimation}


\section{Accuracy Evaluation}
% choose special use case to evaluate

\section{Comparison with Other Approach}