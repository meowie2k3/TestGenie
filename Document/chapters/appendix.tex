\begin{lstlisting}[language=Python, caption={$\texttt{Project}$ class.}, label={lst:1}]
  import os
  import subprocess
  
  class Project:
      _framework = ''
      def __init__(self, git_url):
          self._git_url = git_url
          self._name = git_url.split('/')[-1]
          # print('Project name: ', self._name)
          if self._name.endswith('.git'):
              self._name = self._name[:-4]
          
          # check if project already cloned
          if os.path.exists(projectDir + '/' + self._name):
              return
          else:
              self._clone(git_url)
      
      def _clone(self, git_url):
          # clone the git repository to the project directory
          try:
              # if Project folder not exist, create it
              if not os.path.exists(projectDir):
                  os.makedirs(projectDir)
              return subprocess.check_output(['git', 'clone', git_url, projectDir + '/' + self._name], universal_newlines=True)
          except subprocess.CalledProcessError as e:
              raise Exception(f'Error cloning project: {e}')
      
      def recognizeProjectFramework(self) -> str:
          # TODO: Implement project framework recognition
          return 'flutter'
          pass
      
      def _setFramework(self, framework) -> None:
          self._framework = framework
          
      def getFramework(self) -> str:
          return self._framework
      
      def getName(self) -> str:
          return self._name
      
      def getPath(self) -> str:
          return projectDir + '/' + self._name
          
      def getFileContent(self, fileDir: str) -> str:
          """_summary_
  
          Args:
              fileDir (str): file directory relative to project directory
  
          Returns:
              str: file content
          """
          with open(os.path.join(projectDir, self.getName(), fileDir), 'r') as f:
              return f.read()
\end{lstlisting}

\begin{lstlisting}[language=Python, caption={$\texttt{Flutter}$ class - subclass of Project.}, label={lst:2}]
from ProjectManager import Project, projectDir, os, subprocess, sdkDir

sdkDir = os.path.join(sdkDir, 'flutter')

class Flutter(Project): # Inherit from Project class
    
    def __init__(self, git_url):
        super().__init__(git_url)
        self._setFramework('Flutter')
        self._checkSDK()
        self._flutterPubGet()
        self._addTestDependency()
        self.yaml_name = self._getYamlName()
        # self._createSampleProject('sample')
        
    def _runFlutterCLI(self, args, isRaiseException=False) -> tuple:
        prjDir = os.path.join(projectDir, self.getName())
        flutterBatDir = os.path.join(sdkDir, 'bin', 'flutter')

        cmd = [flutterBatDir]
        # args handling
        # if args is a string that have space, convert it to list
        if isinstance(args, str) and ' ' in args:
            args = args.split()
        if isinstance(args, list):
            cmd.extend(args)
            
        # run cmd via subprocess
        try:
            process = subprocess.Popen(cmd, cwd=prjDir, stdout=subprocess.PIPE, stderr=subprocess.PIPE, universal_newlines=True, encoding='utf-8', shell=True)
            stdout, stderr = process.communicate()
            if process.returncode != 0 and isRaiseException:
                raise Exception(f'Error running flutter command: {stderr}')
            return stdout, stderr
        except subprocess.CalledProcessError as e:
            if isRaiseException:
                raise Exception(f'Error running flutter command: {e}')
            return e.__dict__, e.args
    
    def _checkSDK(self) -> None:
        # Check if flutter sdk is installed
        if not os.path.exists(sdkDir):
            print('Flutter SDK not found')
            return
        # run sdk from sdkDir
        try:
            self._runFlutterCLI('--version', isRaiseException=True)
        except subprocess.CalledProcessError as e:
            raise Exception(f'Error checking flutter sdk: {e}')
        
        # print(result)
    
    def _getYamlName(self) -> str:

        yamlContent = self.getFileContent('pubspec.yaml')
        # print(yamlContent)
        # first line should define the name of the project: "name: ....."
        return yamlContent.split('\n')[0].split('name: ')[1].strip()

    
    # function for testing only. Do not use in production
    def _createSampleProject(self, prjName) -> str:
        try:
            # cannot use _runFlutterCLI because no project directory yet
            # result = self._runFlutterCLI(['create', prjName], isRaiseException=True)
            result = subprocess.check_output([os.path.join(sdkDir, 'bin', 'flutter'), 'create', prjName],cwd=projectDir, universal_newlines=True, encoding='utf-8',  shell=True)
            
        except subprocess.CalledProcessError as e:
            raise Exception(f'Error creating flutter project: {e}')
        return result
    
    def _flutterPubGet(self) -> None:
        # prjDir = os.path.join(projectDir, self.getName())
        # flutterBatDir = os.path.join(sdkDir, 'bin', 'flutter.bat')
        
        try:
            # result = subprocess.check_output([flutterBatDir, 'pub', 'get'], cwd=prjDir, universal_newlines=True)
            self._runFlutterCLI(['pub', 'get', '--no-example'], isRaiseException=True)
        except subprocess.CalledProcessError as e:
            raise Exception(f'Error running flutter pub get: {e}')
        
        # print(result)
        
    def _addTestDependency(self) -> None:
        # run pub add test
        try:
            self._runFlutterCLI(['pub', 'add', 'test'], isRaiseException=True)
        except subprocess.CalledProcessError as e:
            raise Exception(f'Error adding test dependency: {e}')
        # print(result)
    
    def create_test(self, filename, content, isOverWrite = False) -> None:
        # create test file in the test directory
        # check if test directory exists
        testDir = os.path.join(projectDir, self.getName(), 'test')
        if not os.path.exists(testDir):
            os.makedirs(testDir)
        # check if file exists
        fileDir = os.path.join(testDir, filename)
        if os.path.exists(fileDir) and not isOverWrite:
            raise Exception(f'File {fileDir} already exists')
        # create file
        with open(fileDir, 'w') as f:
            f.write(content)
            
    def get_test_content(self, filename) -> str:
        # use getFileContent to get the content of the test file
        testDir = os.path.join(projectDir, self.getName(), 'test')
        fileDir = os.path.join(testDir, filename)
        if not os.path.exists(fileDir):
            raise Exception(f'File {fileDir} does not exist')
        return self.getFileContent(fileDir)
    
    # return tuple (result, error)
    def run_test(self, filename) -> tuple:
        fileDir = os.path.join('test', filename)
        try:
            result = self._runFlutterCLI(['test', fileDir])
        except subprocess.CalledProcessError as e:
            raise Exception(f'Error running flutter test: {e}')
        return result
        pass
    
    def validate(self) -> str:
        # run all tests in the test directory
        testDir = os.path.join(projectDir, self.getName(), 'test')
        for file in os.listdir(testDir):
            if file.endswith('.dart'):
                result, err = self.run_test(file)
                if err:
                    return err
                
        return ''
    
    def getListSourceFiles(self) -> list[str]:
            """_summary_

            Returns:
                list[str]: list of source files in the project relative to project directory
            """
            prjDir = os.path.join(projectDir, self.getName())
            libDir = os.path.join(prjDir, 'lib') 
            sourceFiles = []
            
            # find main.dart first
            if os.path.exists(os.path.join(libDir, 'main.dart')):
                sourceFiles.append(os.path.relpath(os.path.join(libDir, 'main.dart'), prjDir))
            
            for root, dirs, files in os.walk(libDir):
                for file in files:
                    if file.endswith('.dart') and os.path.relpath(os.path.join(root, file), prjDir) not in sourceFiles:
                        sourceFiles.append(os.path.relpath(os.path.join(root, file), prjDir))
                        
            return sourceFiles
    
    def __str__(self) -> str:
        return f'Flutter project {self.getName()} created from {self._git_url}'
    
    pass
 \end{lstlisting}

\begin{lstlisting}[language=Python, caption={$\texttt{DependencyDiagram}$ class.}, label={lst:3}]
    from ProjectManager import Project
    from .Flutter import FlutterAnalyzeStrategy
    from .AI_Agent import AI_Agent
    
    class DependencyDiagram:
        
        blocks = []
        connections = []
        
        def __init__(self, project: Project) -> None:
            self.project = project
            self._generateDiagram()
            self.ai_agent = AI_Agent()
            self._getPredictions()
        
        def _generateDiagram(self) -> None:
            # Analyze project abstractly to project's framework
            framework = self.project.getFramework()
            functionName = framework + 'AnalyzeStrategy'
            if functionName in globals():
                globals()[functionName](self)
            else:
                raise Exception('Framework not supported')
            
        def _getPredictions(self) -> None:
            for block in self.blocks:
                block.setPrediction(self.ai_agent.generate_BLA_prediction(source_code=block.getContentNoComment(), chat_history=[]))
            
        def __str__(self) -> str:
            """_summary_
    
            Returns:
                str: project name, list of blocks and connections in the diagram
            """
            res = f'Project: {self.project.getName()}\n'
            res += 'Blocks:\n'
            for block in self.blocks:
                res += f'{block.name} - {block.type}\n'
            res += 'Connections:\n'
            for connection in self.connections:
                res += f'{connection.head.name} -> {connection.tail.name} - {connection.type}\n'
            return res
            
\end{lstlisting}
