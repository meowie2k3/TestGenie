\begin{lstlisting}[language=Python, caption={$\texttt{Project}$ class.}, label={lst:1}]
  import os
  import subprocess
  
  class Project:

      _framework = ''
      def __init__(self, git_url):
          self._git_url = git_url
          self._name = git_url.split('/')[-1]
          # print('Project name: ', self._name)
          if self._name.endswith('.git'):
              self._name = self._name[:-4]
          
          # check if project already cloned
          if os.path.exists(projectDir + '/' + self._name):
              return
          else:
              self._clone(git_url)
      
      def _clone(self, git_url):
          # clone the git repository to the project directory
          try:
              # if Project folder not exist, create it
              if not os.path.exists(projectDir):
                  os.makedirs(projectDir)
              return subprocess.check_output(['git', 'clone', git_url, projectDir + '/' + self._name], universal_newlines=True)
          except subprocess.CalledProcessError as e:
              raise Exception(f'Error cloning project: {e}')
      
      def recognizeProjectFramework(self) -> str:
          # TODO: Implement project framework recognition
          return 'flutter'
          pass
      
      def _setFramework(self, framework) -> None:
          self._framework = framework
          
      def getFramework(self) -> str:
          return self._framework
          
      def getName(self) -> str:
          return self._name
          
      def getPath(self) -> str:
          return projectDir + '/' + self._name
          
      def getFileContent(self, fileDir: str) -> str:
          """_summary_
  
          Args:
              fileDir (str): file directory relative to project directory
  
          Returns:
              str: file content
          """
          with open(os.path.join(projectDir, self.getName(), fileDir), 'r') as f:
              return f.read()
\end{lstlisting}

\begin{lstlisting}[language=Python, caption={$\texttt{Flutter}$ class - subclass of Project.}, label={lst:2}]
from ProjectManager import Project, projectDir, os, subprocess, sdkDir

sdkDir = os.path.join(sdkDir, 'flutter')

class Flutter(Project): # Inherit from Project class
    
    def __init__(self, git_url):
        super().__init__(git_url)
        self._setFramework('Flutter')
        self._checkSDK()
        self._flutterPubGet()
        self._addTestDependency()
        self.yaml_name = self._getYamlName()
        
    def _runFlutterCLI(self, args, isRaiseException=False) -> tuple:
        # Set up command execution environment
        # Execute Flutter command and handle results
        prjDir = os.path.join(projectDir, self.getName())
        flutterBatDir = os.path.join(sdkDir, 'bin', 'flutter')

        cmd = [flutterBatDir]
        # args handling
        # if args is a string that have space, convert it to list
        if isinstance(args, str) and ' ' in args:
            args = args.split()
        if isinstance(args, list):
            cmd.extend(args)
            
        # run cmd via subprocess
        try:
            process = subprocess.Popen(cmd, cwd=prjDir, stdout=subprocess.PIPE, stderr=subprocess.PIPE, universal_newlines=True, encoding='utf-8', shell=True)
            stdout, stderr = process.communicate()
            if process.returncode != 0 and isRaiseException:
                raise Exception(f'Error running flutter command: {stderr}')
            return stdout, stderr
        except subprocess.CalledProcessError as e:
            if isRaiseException:
                raise Exception(f'Error running flutter command: {e}')
            return e.__dict__, e.args
    
    def _checkSDK(self) -> None:
        # Check if flutter sdk is installed
        if not os.path.exists(sdkDir):
            print('Flutter SDK not found')
            return
        # run sdk from sdkDir
        try:
            self._runFlutterCLI('--version', isRaiseException=True)
        except subprocess.CalledProcessError as e:
            raise Exception(f'Error checking flutter sdk: {e}')
        
        # print(result)
    
    def _getYamlName(self) -> str:

        yamlContent = self.getFileContent('pubspec.yaml')
        # print(yamlContent)
        # first line should define the name of the project: "name: ....."
        return yamlContent.split('\n')[0].split('name: ')[1].strip()

    
    # function for testing only. Do not use in production
    def _createSampleProject(self, prjName) -> str:
        try:
            # cannot use _runFlutterCLI because no project directory yet
            # result = self._runFlutterCLI(['create', prjName], isRaiseException=True)
            result = subprocess.check_output([os.path.join(sdkDir, 'bin', 'flutter'), 'create', prjName],cwd=projectDir, universal_newlines=True, encoding='utf-8',  shell=True)
            
        except subprocess.CalledProcessError as e:
            raise Exception(f'Error creating flutter project: {e}')
        return result
    
    def _flutterPubGet(self) -> None:
        # prjDir = os.path.join(projectDir, self.getName())
        # flutterBatDir = os.path.join(sdkDir, 'bin', 'flutter.bat')
        
        try:
            # result = subprocess.check_output([flutterBatDir, 'pub', 'get'], cwd=prjDir, universal_newlines=True)
            self._runFlutterCLI(['pub', 'get', '--no-example'], isRaiseException=True)
        except subprocess.CalledProcessError as e:
            raise Exception(f'Error running flutter pub get: {e}')
        
        # print(result)
        
    def _addTestDependency(self) -> None:
        # run pub add test
        try:
            self._runFlutterCLI(['pub', 'add', 'test'], isRaiseException=True)
        except subprocess.CalledProcessError as e:
            raise Exception(f'Error adding test dependency: {e}')
        # print(result)
    
    def create_test(self, filename, content, isOverWrite = False) -> None:
        # create test file in the test directory
        # check if test directory exists
        testDir = os.path.join(projectDir, self.getName(), 'test')
        if not os.path.exists(testDir):
            os.makedirs(testDir)
        # check if file exists
        fileDir = os.path.join(testDir, filename)
        if os.path.exists(fileDir) and not isOverWrite:
            raise Exception(f'File {fileDir} already exists')
        # create file
        with open(fileDir, 'w') as f:
            f.write(content)
            
    def get_test_content(self, filename) -> str:
        # use getFileContent to get the content of the test file
        testDir = os.path.join(projectDir, self.getName(), 'test')
        fileDir = os.path.join(testDir, filename)
        if not os.path.exists(fileDir):
            raise Exception(f'File {fileDir} does not exist')
        return self.getFileContent(fileDir)
    
    # return tuple (result, error)
    def run_test(self, filename) -> tuple:
        fileDir = os.path.join('test', filename)
        try:
            result = self._runFlutterCLI(['test', fileDir])
        except subprocess.CalledProcessError as e:
            raise Exception(f'Error running flutter test: {e}')
        return result
        pass
    
    def validate(self) -> str:
        # run all tests in the test directory
        testDir = os.path.join(projectDir, self.getName(), 'test')
        for file in os.listdir(testDir):
            if file.endswith('.dart'):
                result, err = self.run_test(file)
                if err:
                    return err
                
        return ''
    
    def getListSourceFiles(self) -> list[str]:
        """Returns list of source files in the project relative to project directory"""
        prjDir = os.path.join(projectDir, self.getName())
        libDir = os.path.join(prjDir, 'lib') 
        sourceFiles = []
        
        # Find main.dart first
        if os.path.exists(os.path.join(libDir, 'main.dart')):
            sourceFiles.append(os.path.relpath(os.path.join(libDir, 'main.dart'), prjDir))
        
        # Walk through directory to find all Dart files
        for root, dirs, files in os.walk(libDir):
            for file in files:
                if file.endswith('.dart') and os.path.relpath(os.path.join(root, file), prjDir) not in sourceFiles:
                    sourceFiles.append(os.path.relpath(os.path.join(root, file), prjDir))
                    
        return sourceFiles
    
    def __str__(self) -> str:
        return f'Flutter project {self.getName()} created from {self._git_url}'
    
    pass
 \end{lstlisting}

\begin{lstlisting}[language=Python, caption={$\texttt{DependencyDiagram}$ class.}, label={lst:3}]
    from ProjectManager import Project
    from .Flutter import FlutterAnalyzeStrategy
    from .AI_Agent import AI_Agent
    
    class DependencyDiagram:
        
        blocks = []
        connections = []
        
        def __init__(self, project: Project) -> None:
            self.project = project
            self._generateDiagram()
            self.ai_agent = AI_Agent()
            self._getPredictions()
        
        def _generateDiagram(self) -> None:
            framework = self.project.getFramework()
            functionName = framework + 'AnalyzeStrategy'
            if functionName in globals():
                globals()[functionName](self)
            else:
                raise Exception('Framework not supported')
            
        def _getPredictions(self) -> None:
            for block in self.blocks:
                block.setPrediction(self.ai_agent.generate_BLA_prediction(source_code=block.getContentNoComment(), chat_history=[]))
\end{lstlisting}

\begin{lstlisting}[language=Python, caption={$\texttt{Block}$ class.}, label={lst:4}]
    class Block:
    def __init__(self, name: str, content: str, type: str) -> None:
        self.name = name
        self.content = content
        self.type = type

    def getContentNoComment(self) -> str:
        # Remove single-line and multi-line comments from code
        content = self.content
        res = ''
        i = 0
        isCommentSingleLine = False
        isCommentMultiLine = False
        
        # Iterate through content character by character
        # Track comment blocks and exclude them from result
        while i < len(self.content)-1:
            # if \n, reset isCommentSingleLine
            if content[i] == '\n':
                isCommentSingleLine = False
            if content[i] == '/' and content[i+1] == '*':
                isCommentMultiLine = True
            if content[i] == '/' and content[i+1] == '/':
                isCommentSingleLine = True
            if not isCommentSingleLine and not isCommentMultiLine:
                res += content[i]
            if content[i] == '*' and content[i+1] == '/':
                isCommentMultiLine = False
                i+=1
            i+=1
        
        # delete all empty lines
        res = '\n'.join([line for line in res.split('\n') if line.strip() != ''])
        
        return res
    
    def setPrediction(self, prediction: str) -> None:
        self.prediction = prediction

    def getPrediction(self) -> str:
        return self.prediction
\end{lstlisting}

\begin{lstlisting}[language=Python, caption={$\texttt{BlockType}$ class (Enumerate).}, label={lst:5}]
    class BlockType:
        FILE = 'File'
        CLASS = 'Class'
        ABSTRACT_CLASS = 'AbstractClass'
        ENUM = 'Enum'
        GLOBAL_VAR = 'GlobalVar'
        FUNCTION = 'Function'
        CLASS_CONSTRUCTOR = 'ClassConstructor'
        CLASS_FUNCTION = 'ClassFunction'
        CLASS_ATTRIBUTE = 'ClassAttribute'
\end{lstlisting}

\begin{lstlisting}[language=Python, caption={$\texttt{Connection}$ class.}, label={lst:6}]
    class Connection:
    def __init__(self, head: Block, tail: Block, type: str):
        self.head = head
        self.tail = tail
        self.type = type
\end{lstlisting}

\begin{lstlisting}[language=Python, caption={$\texttt{ConnectionType}$ class (Enumerate).}, label={lst:7}]
    class ConnectionType:
        EXTEND = 'Extend'
        IMPLEMENT = 'Implement'
        CONTAIN = 'Contain'
        EXTEND = 'Extend'
        USE = 'Use'
        CALL = 'Call'
        IMPORT = 'Import'
\end{lstlisting}
 
\begin{lstlisting}[language=Python, caption={$\texttt{FlutterAnalyzeStrategy}$ function.}, label={lst:8}]
    def FlutterAnalyzeStrategy(diagram) -> None:
    # Get list of source files from the project
    fileList = diagram.project.getListSourceFiles()
    
    # Create a block for main.dart file first
    mainfileDir = fileList[0]
    mainFileContent = diagram.project.getFileContent(mainfileDir)
    mainfileDir = mainfileDir.replace('\\','/')
    mainBlock = Block(mainfileDir, mainFileContent, BlockType.FILE)
    diagram.blocks.append(mainBlock)
    
    # Run the analysis phases in sequence
    ImportAnalyzer(diagram, diagram.blocks[0])  # Analyze imports between files
    ContainAnalyzer(diagram, diagram.blocks[0]) # Identify code blocks and containment
    CallAnalyzer(diagram, diagram.blocks[0])    # Analyze call relationships
\end{lstlisting}

\begin{lstlisting}[language=Python, caption={$\texttt{ImportAnalyzer}$ function.}, label={lst:9}]
def ImportAnalyzer(diagram, block):
    # If current block is a File, analyze its import statements
    if block.type == 'File':
        # Extract import lines from the file content
        importLines = [line.strip() for line in block.content.split('\n') 
                      if line.strip().startswith('import')]
        
        blocks = []
        for line in importLines:
            # Extract directory path from import statement
            directory = line.split(' ')[1].replace(';', '')
            directory = directory[1:-1]  # Remove quotes
            
            # Handle different import types:
            # 1. Package imports from dependencies
            # 2. Package imports from project
            # 3. Relative path imports
            
            # Create blocks for imported files and add connections
            if directory.startswith('package:'):
                # import from other package, import from project
                prjName = diagram.project.yaml_name
                if directory.startswith(f'package:{prjName}'):
                    # import from project
                    # create block for this file and connection
                    fileDir = directory.split(f'package:{prjName}')[1]
                    fileDir = 'lib' + fileDir
                    fileContent = diagram.project.getFileContent(fileDir)
                    # if fileDir is not in Diagram.blocks
                    if not any(block.name == fileDir for block in diagram.blocks):
                        blocks.append(Block(fileDir, fileContent, BlockType.FILE))
                    else: diagram.connections.append(Connection(block, [b for b in diagram.blocks if b.name == fileDir][0], ConnectionType.IMPORT))
            else:
                # import as relative path
                currentDir = block.name #ex: lib/main.dart
                currentDir = currentDir.split('/')
                currentDir.pop()
                currentDir = '/'.join(currentDir)
                combineDir = os.path.normpath(os.path.join(currentDir, directory))
                # print(combineDir)
                fileContent = diagram.project.getFileContent(combineDir)
                if combineDir not in [block.name for block in diagram.blocks]:
                    # turn \ into /
                    combineDir = combineDir.replace('\\','/')
                    blocks.append(Block(combineDir, fileContent, BlockType.FILE))
                else: diagram.connections.append(Connection(block, [b for b in diagram.blocks if b.name == combineDir][0], ConnectionType.IMPORT))
        
        # Process each new imported file recursively
        for b in blocks:
            diagram.blocks.append(b)
            diagram.connections.append(Connection(block, b, ConnectionType.IMPORT))
            ImportAnalyzer(diagram, b)
\end{lstlisting}

\begin{lstlisting}[language=pseudocode, caption={$\texttt{ContainAnalyzer}$ function (Pseudocode).}, label={lst:10}]
FUNCTION ContainAnalyzer(diagram, block, visited):
    IF block IN visited THEN
        RETURN
    END IF
    
    ADD block TO visited
    
    IF block.type IS FILE OR CLASS OR ABSTRACT_CLASS THEN
        content = block.getContentNoComment()
        blocks = []
        
        IF block.type IS FILE THEN
            # Analyze class declarations
            FOR EACH line IN content.lines
                IF line CONTAINS "class " OR "abstract class " THEN
                    Extract class name and content
                    Create appropriate Block object
                    Add to blocks list
                END IF
            END FOR
            
            # Analyze enum declarations
            FOR EACH enum declaration IN content
                Extract enum name and content
                Create Block with type ENUM
                Add to blocks list
            END FOR
            
            # Analyze standalone functions and global variables
            Extract functions and globals from remaining content
            Add to blocks list
        END IF
        
        IF block.type IS CLASS OR ABSTRACT_CLASS THEN
            Extract class methods and attributes
            Add to blocks list
        END IF
        
        # Process each identified block
        FOR EACH b IN blocks
            diagram.blocks.append(b)
            diagram.connections.append(Connection(block, b, ConnectionType.CONTAIN))
            ContainAnalyzer(diagram, b, visited)
        END FOR
    END IF
    
    # Continue analysis with connected blocks
    connectedBlocks = [c.tail FOR EACH c IN diagram.connections 
                      WHERE c.head IS block AND c.tail NOT IN visited]
    
    FOR EACH b IN connectedBlocks
        ContainAnalyzer(diagram, b, visited)
    END FOR
END FUNCTION
\end{lstlisting}

\begin{lstlisting}[language=pseudocode, caption={$\texttt{CallAnalyzer}$ function (Pseudocode).}, label={lst:11}]
FUNCTION CallAnalyzer(diagram, block, visited):
    IF block IN visited THEN
        RETURN
    END IF
    
    ADD block TO visited
    
    IF block.type IS FILE THEN
        # Find connected files (imported files)
        connectedFiles = [conn.tail FOR EACH conn IN diagram.connections 
                         WHERE conn.head IS block AND conn.type IS ConnectionType.IMPORT]
        
        FOR EACH file IN connectedFiles
            # Find blocks defined in both files
            connectedBlocks = [blocks defined in file]
            currentBlocks = [blocks defined in current file]
            
            # Analyze call relationships between blocks
            _CallAnalyzer(diagram, currentBlocks, connectedBlocks, visited)
            
            # Continue analysis with imported files
            CallAnalyzer(diagram, file, visited)
        END FOR
    END IF
END FUNCTION

FUNCTION _CallAnalyzer(diagram, thisFile, callables, visited):
    # Combine blocks from current file with callable blocks
    callables.extend(thisFile)
    
    FOR EACH block IN thisFile:
        IF block IN visited THEN
            CONTINUE
        END IF
        
        IF block.type IS CLASS OR ABSTRACT_CLASS THEN
            # Analyze inheritance relationships
            Identify parent classes and create EXTEND connections
            
            # Recursively process class contents
            innerBlocks = [contained blocks]
            _CallAnalyzer(diagram, innerBlocks, callables, visited)
        ELSE
            # Extract relevant content (function body, attribute initializers)
            content = [process content based on block type]
            
            # Look for references to other blocks
            FOR EACH callable name and block:
                IF name appears in content THEN
                    diagram.connections.append(Connection(block, callable, ConnectionType.CALL))
                END IF
            END FOR
        END IF
    END FOR
END FUNCTION
\end{lstlisting}

\begin{lstlisting}[language=pseudocode, caption={$\texttt{AI\_Agent}$ class (Pseudocode).}, label={lst:12}]
CLASS AI_Agent:
    CONSTRUCTOR():
        # Load environment variables and configuration
        Load API credentials and model settings
        
        # Initialize LLM model and embeddings
        self.model = Initialize ChatOpenAI with temperature=0
        self.embeddings = Initialize OpenAIEmbeddings
        
        # Setup vector stores for document retrieval
        self.store_names = {
            "flutter_tutorial": "flutter_tutorial.pdf",
        }
        
        FOR EACH store_name, doc_name IN self.store_names:
            IF vector store doesn't exist THEN
                Load and split document
                Create vector store
            END IF
        END FOR
        
        # Initialize retrievers for each vector store
        Create similarity_score_threshold retrievers
        
        # Initialize agent
        self._agent_init()
    
    METHOD generate_BLA_prediction(source_code, chat_history):
        # First analyze code with the agent
        response = self.agent_executor.invoke(source_code, chat_history)
        
        # Structure the output with proper formatting
        structured_prompt = Create prompt for formatting the analysis
        structured_response = self.model.invoke(structured_prompt)
        
        RETURN structured_response.content
    
    METHOD _agent_init():
        # Create context-aware prompt template
        Create contextualize_q_system_prompt
        
        # Create history-aware retrievers
        Create retrievers that incorporate chat history
        
        # Create system prompt for business logic analysis
        Create detailed bla_system_prompt
        
        # Create chain for document processing
        Create bla_chain for processing documents
        
        # Create RAG chains for each retriever
        Create retrieval_chain for each retriever
        
        # Create tools for agent
        Create tool for each store_name
        
        # Initialize agent
        Create react agent with tools
        Initialize agent_executor
END CLASS
\end{lstlisting}

\begin{lstlisting}[language=Python, caption={$\texttt{Sample .env}$ file.}, label={lst:13}]
    OPENAI_API_KEY=sk-this-key-is-just-a-placeholder
    LANGCHAIN_API_KEY=sk-this-key-is-just-a-placeholder
    LANGCHAIN_PROJECT=TestGenie
    
    BASE_URL=
    BLA_LLM_MODEL=
    TG_LLM_MODEL=
    
    EMBED_MODEL=
\end{lstlisting}

\begin{lstlisting}[language=pseudocode, caption={$\texttt{Test\_Generator}$ class (Pseudocode).}, label={lst:14}]
CLASS Test_Generator:
    CONSTRUCTOR():
        # Load environment and initialize models
        Load environment variables
        Initialize LLM model and embeddings
        
        # Set up vector stores for retrieval
        Load and initialize vector stores for testing documentation
        Create similarity-based retrievers
        
        # Initialize error handling infrastructure
        self.error_fix_cache = {}
        self.attempted_fixes_for_error = {}
        self.max_error_fix_attempts = 3
    
    METHOD generate_test_case(package_name, code_location, function_name_and_arguments, prediction):
        # Structure prediction if needed
        IF "TESTING SCENARIOS:" NOT IN prediction THEN
            Create structured prediction
        END IF
        
        # Generate clean test code
        raw_output = self._generate_clean_test(parameters)
        
        # Clean up markdown or formatting
        cleaned_output = self._clean_code_output(raw_output)
        
        RETURN cleaned_output
    
    METHOD fix_generated_code(error_message, current_test_code, prediction):
        # Create unique hash for this error
        error_hash = self._generate_error_hash(error_message, current_test_code)
        
        # Check cache for previously seen error
        IF error_hash IN self.error_fix_cache THEN
            RETURN cached fix
        END IF
        
        # Track fix attempts
        Increment attempt counter
        
        # Apply different strategies based on attempt number
        IF first attempt THEN
            fixed_code = self._standard_ai_fix(error_message, current_test_code)
        ELSE IF second attempt THEN
            online_solutions = self._search_for_error_solutions(error_message)
            fixed_code = self._ai_fix_with_online_knowledge(error_message, current_test_code, online_solutions)
        ELSE
            fixed_code = self._comprehensive_repair_attempt(error_message, current_test_code, prediction)
        END IF
        
        # Apply additional specific fixes
        fixed_code = self._apply_specific_fixes(fixed_code, error_message)
        
        # Cache the fix
        self.error_fix_cache[error_hash] = fixed_code
        
        RETURN fixed_code
END CLASS
\end{lstlisting}

\begin{lstlisting}[language=Python, caption={$\texttt{main.py}$ file.}, label={lst:15}]
    frameworkMap = {
        'flutter': Flutter
    }
    
    def getDBMS(git_url) -> DBMS:
        project = Project(git_url)
        framework = project.recognizeProjectFramework()
        
        if framework in frameworkMap:
            project = frameworkMap[framework](git_url)
            
        dbms = DBMS(project)
        
        return dbms
    
    app = Flask(__name__)
    CORS(app)
    
    # API endpoints implementation
    @app.route('/createProject', methods=['POST'])
    def createProject():
        if not request.json or not 'git_url' in request.json:
            return jsonify({'message': 'Invalid request'})
        git_url = request.json['git_url']
        project = Project(git_url)
        # print(project)
        return jsonify({'message': f'{project}'})
    
    @app.route('/getDiagram', methods=['POST'])
    def getDiagram():
        if not request.json or not 'git_url' in request.json:
            return jsonify({'message': 'Invalid request'})
        git_url = request.json['git_url']
        
        dbms = getDBMS(git_url)
        
        diagram = dbms.getJsonDiagram()
        return jsonify(diagram)
    
    @app.route('/getDiagram', methods=['OPTIONS'])
    def getDiagramOptions():
        print(request.json)
        print("wrong method")
        return jsonify({'message': 'Options request'})
    
    @app.route('/getBlockContent', methods=['POST'])
    def getBlockContent():
        if not request.json or not 'git_url' in request.json or not 'block_id' in request.json:
            return jsonify({'message': 'Invalid request'})
        git_url = request.json['git_url']
        blockId = request.json['block_id']
        
        dbms = getDBMS(git_url)
        blockContent = dbms.getBlockContent(blockId)
        return jsonify(blockContent)
        
    @app.route('/getBlockPrediction', methods=['POST'])
    def getBlockPrediction():
        if not request.json or not 'git_url' in request.json or not 'block_id' in request.json:
            return jsonify({'message': 'Invalid request'})
        git_url = request.json['git_url']
        blockId = request.json['block_id']
        
        dbms = getDBMS(git_url)
        blockPrediction = dbms.getBlockPrediction(blockId)
        return jsonify(blockPrediction)
    
    @app.route('/getBlockDetail', methods=['POST'])
    def getBlockDetail():
        if not request.json or not 'git_url' in request.json or not 'block_id' in request.json:
            return jsonify({'message': 'Invalid request'})
        git_url = request.json['git_url']
        blockId = request.json['block_id']
        
        dbms = getDBMS(git_url)
        # {
           # 'content': blockContent,
             # 'prediction': blockPrediction, 
        # }
        content = dbms.getBlockContent(blockId)
        prediction = dbms.getBlockPrediction(blockId)
        try: 
            test_file_content = dbms.project.get_test_content ('block_' + str(blockId) + '_test.dart')
        except Exception as e:
            test_file_content = ''
            
        return jsonify({
            'content': content,
            'prediction': prediction,
            'test_file_content': test_file_content,
        })
    
    # dont know why this is needed
    @app.route('/getBlockDetail', methods=['OPTIONS'])
    def getBlockDetailOptions():
        print(request.json)
        print("wrong method")
        return jsonify({'message': 'Options request'})
    
    @app.route('/updateBlockPrediction', methods=['POST'])
    def updateBlockPrediction():
        if not request.json or not 'git_url' in request.json or not 'block_id' in request.json or not 'prediction' in request.json:
            return jsonify({'message': 'Invalid request'})
        git_url = request.json['git_url']
        blockId = request.json['block_id']
        prediction = request.json['prediction']
        
        dbms = getDBMS(git_url)
        dbms.updateBlockPrediction(blockId, prediction)
        
        return jsonify(
            {
                'message': 'Update success!',
                'code': 200,
                'success': True,
            }
        )
    
    @app.route('/updateBlockPrediction', methods=['OPTIONS'])
    def updateBlockPredictionOptions():
        print(request.json)
        print("wrong method")
        return jsonify({'message': 'Options request'})
        
    @app.route('/generateTest', methods=['POST'])
    def generateTest():
        try:
            if not request.json or not 'git_url' in request.json or not 'block_id' in request.json:
                return jsonify({'message': 'Invalid request'})
            git_url = request.json['git_url']
            blockId = request.json['block_id']
            
            dbms = getDBMS(git_url)
            tg = Test_Generator()
            
            testFileContent = tg.generate_test_case(
                package_name= dbms.project.getName(),
                code_location=dbms.getBlockOriginalFile(blockId),
                function_name_and_arguments=dbms.getBlockName(blockId),
                prediction=dbms.getBlockPrediction(blockId),
            )
            file_name = 'block_' + str(blockId) + '_test.dart'
            
            dbms.project.create_test(
                filename=file_name,
                content=testFileContent,
                isOverWrite=True
            )
            # validation process
            run_result, run_error = dbms.project.run_test(file_name)
            iteration_limit = 5
            while run_error != '' and iteration_limit > 0:
                new_test_content = tg.fix_generated_code(
                    error_message=run_error,
                    current_test_code=testFileContent,
                    prediction=dbms.getBlockPrediction(blockId),
                )
                dbms.project.create_test(
                    filename=file_name,
                    content=new_test_content,
                    isOverWrite=True
                )
                run_result, run_error = dbms.project.run_test(file_name)
                iteration_limit -= 1
            
            return jsonify(
                {
                    'message': 'Test generation success!',
                    'code': 200,
                    'success': True,
                    'test_file_content': testFileContent,
                }
            )
        except Exception as e:
            return jsonify({'message': str(e)})
    
    @app.route('/generateTest', methods=['OPTIONS'])
    def generateTestOptions():
        print(request.json)
        print("wrong method")
        return jsonify({'message': 'Options request'})
        
    if __name__ == '__main__':
        app.run(debug=True)
\end{lstlisting}

\begin{lstlisting}[language=Python, caption={$\texttt{Table}$ class.}, label={lst:16}]
class Table:
    def __init__(self, name: str, columns: dict):
        self.name = name
        self.columns = columns
        
    def getCreateSQL(self):
        sql = f'CREATE TABLE IF NOT EXISTS {self.name} ('
        for column in self.columns:
            sql += f'{column} {self.columns[column]}, '
        sql = sql[:-2] + ')'
        return sql
    
    def getSelectSQL(self,fields: list ,conditions: dict):
        # if conditions is empty, return all
        res = f'SELECT '
        if len(fields) == 0:
            res += '*'
        else:
            for field in fields:
                res += f'{field}, '
            res = res[:-2]
        res += f' FROM {self.name}'
        
        if len(conditions) > 0:
            res += ' WHERE '
            for condition in conditions:
                res += f"{condition} = '{conditions[condition]}' AND "
            res = res[:-4]
            
        return res
        pass
    
    def getInsertSQL(self, values: dict):
        sql = f'INSERT INTO {self.name} ('
        for column in values:
            sql += f'{column}, '
        sql = sql[:-2] + ") VALUES ("
        for column in values:
            sql += f"'{values[column]}', "
        sql = sql[:-2] + ')'
        return sql
    
    def getUpdateSQL(self, values: dict, conditions: dict):
        sql = f"UPDATE {self.name} SET "
        for column in values:
            sql += f"{column} = \'{values[column]}\', "
        sql = sql[:-2] + " WHERE "
        for column in conditions:
            sql += f"{column} = \'{conditions[column]}\' AND "
        sql = sql[:-4]
        return sql
    
\end{lstlisting}

\begin{lstlisting}[language=Python, caption={$\texttt{getTable}$ function - BlockType class.}, label={lst:17}]
    @staticmethod
    def getTable():
        from DBMS.Table import Table
        return Table(
            'BlockType',
            {
                'id': 'INT AUTO_INCREMENT PRIMARY KEY',
                'name': 'VARCHAR(255)'
            }
        )
\end{lstlisting}

\begin{lstlisting}[language=Python, caption={$\texttt{getTable}$ function - Block class.}, label={lst:18}]
    @staticmethod
    def getTable():
        from DBMS.Table import Table
        return Table(
            'Block',
            {
                'id': 'INT AUTO_INCREMENT PRIMARY KEY',
                'name': 'VARCHAR(255)',
                'content': 'TEXT',
                'prediction': 'TEXT',
                'type': 'INT',
                '': 'FOREIGN KEY (type) REFERENCES BlockType(id)'
            }
        )
\end{lstlisting}

\begin{lstlisting}[language=Python, caption={$\texttt{getTable}$ function - ConnectionType class.}, label={lst:19}]
    @staticmethod
    def getTable():
        from DBMS.Table import Table
        return Table(
            'ConnectionType',
            {
                'id': 'INT AUTO_INCREMENT PRIMARY KEY',
                'name': 'VARCHAR(255)'
            }
        )
\end{lstlisting}

\begin{lstlisting}[language=Python, caption={$\texttt{getTable}$ function - Connection class.}, label={lst:20}]
    @staticmethod
    def getTable():
        from DBMS.Table import Table
        return Table(
            'Connection',
            {
                'id': 'INT AUTO_INCREMENT PRIMARY KEY',
                'head': 'INT',
                'tail': 'INT',
                'type': 'INT',
                '': 'FOREIGN KEY (head) REFERENCES Block(id)',
                '':'FOREIGN KEY (tail) REFERENCES Block(id)',
                '': 'FOREIGN KEY (type) REFERENCES ConnectionType(id)'
            }
        )
\end{lstlisting}

\begin{lstlisting}[language=pseudocode, caption={$\texttt{DBMS}$ class (Pseudocode).}, label={lst:21}]
CLASS DBMS:
    CONSTRUCTOR(project):
        self.project = project
        
        # Initialize database if needed
        IF NOT self._isDBinit() THEN
            self._initDB()
        END IF
        
        # Insert project if not exists
        IF NOT self._isProjectExistInDB() THEN
            self._insertProject()
        END IF
    
    METHOD getJsonDiagram():
        # Retrieve diagram data from database
        Query blocks and connections from DB
        
        # Format data as JSON structure
        Create project, blocks, and connections structure
        Convert block and connection types to string names
        
        RETURN formatted diagram
    
    METHOD getBlockName(blockId):
        Query block name from database
        RETURN result
    
    METHOD getBlockContent(blockId):
        Query block content from database
        RETURN result
    
    METHOD getBlockPrediction(blockId):
        Query block prediction from database
        RETURN result
    
    METHOD getBlockOriginalFile(blockId):
        # Backtrack through connections to find containing file
        WHILE current block is not a File:
            Query for parent block
            Update current block
        END WHILE
        
        Extract file path relative to project
        RETURN path
    
    METHOD updateBlockPrediction(blockId, prediction):
        Escape special characters in prediction
        Update prediction in database
    
    METHOD _initDB():
        # Create required tables
        Create Project table
        Create Block table
        Create Connection table
        Insert enum values
    
    METHOD _insertProject():
        # Insert project record
        Insert project into database
        
        # Create and store dependency diagram
        Create DependencyDiagram
        Convert blocks and connections to database records
    
    METHOD _mapBlocksIntoDB(blocks):
        FOR EACH block IN blocks:
            Insert block properties into database
        END FOR
    
    METHOD _mapConnectionsIntoDB(connections):
        FOR EACH connection IN connections:
            Insert connection with block IDs into database
        END FOR
END CLASS
\end{lstlisting}