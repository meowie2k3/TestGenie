%% TODO: Add a small paragraph to tell what this chapter is about
This chapter delves into ...

%% TODO: Make sure to use \textbf or \textit for highlighting keywords, and \cite{} to cite the corresponding quotations
\section{Unit test generator}

\hspace{0.5cm} \textbf{LLMs approach compared to fomulated approach.} To accurately give test case with correct syntax, I have researched some techniques that can handle different frameworks with just one centrialized system. There is a research that compares the performance of some common approaches including search-based, constraint-based and random-based. Tests generated by these methods frequently lack meaningful structure or descriptive naming conventions, making them difficult for developers to interpret and modify ~\cite{UnitTest}. This limitation can hinder their practical usability, particularly in dynamic and iterative development environments.

\hspace{0.5cm} In contrast, test case generation using Large Language Models (LLMs) offers a more intuitive and human-aligned approach~\cite{UnitTest}. LLMs, trained on vast amounts of programming-related data, possess the capability to generate test cases that not only adhere to syntactical correctness but also align closely with human developers' intentions and coding practices. This alignment results in unit tests that are more readable, contextually relevant, and easier to understand. Developers can quickly adjust and refine these tests as needed, enhancing their utility in real-world scenarios.

\hspace{0.5cm} Moreover, the flexibility of LLMs enables them to adapt seamlessly to various programming languages and frameworks, providing a centralized solution for diverse development ecosystems. While traditional approaches may produce marginally higher percentages of technically correct test cases, they often lack the usability and adaptability that LLM-based methods provide. As a result, services leveraging LLMs for test generation consistently receive more favorable user feedback due to their focus on developer experience, ease of use, and alignment with real-world development workflows.

\hspace{0.5cm} \textbf{Disadvantages of  LLMs.} One of the most significant challenges is their propensity to generate hallucinations, where the model produces incorrect or fabricated outputs that lack grounding in factual data. This issue is particularly critical in tasks requiring precision, such as author attribution. For instance, research introducing the Simple Hallucination Index (SHI) revealed that even advanced LLMs like Mixtral 8x7B, LLaMA-2-13B, and Gemma-7B suffered from hallucinations, with Mixtral 8x7B achieving an SHI as high as 0.87 for certain datasets ~\cite{LLMLimitations}. These hallucinations undermine the reliability and trustworthiness of LLMs, especially in contexts where factual accuracy is crucial. 

\hspace{0.5cm} Another drawback of LLMs is their lack of transparency in decision-making. These models function as black boxes, providing little insight into the reasoning behind their outputs ~\cite{LLMLimitations}. This opacity complicates the debugging process and limits the ability to verify results, which is particularly problematic in applications requiring a high degree of explainability. Additionally, LLMs are highly dependent on the quality and diversity of their training data. Biases or inaccuracies present in the data can result in outputs that reinforce those biases or produce flawed results. Moreover, while these models excel at generating output based on their training corpus, they often struggle to generalize effectively when faced with novel or unseen cases.



\section{Understanding Business Logic}

\hspace{0.5cm} \textbf{The concept of Business Logic.} An industry's business logic can be seen as a description of a number of basic conditions or circumstances that make up important starting points for understanding an established business and its conditions for change~\cite{BusinessRules}. It encodes the real-world policies, procedures, and processes that govern how data is created, managed, and manipulated in a way that aligns with the objectives of the organization. Business logic acts as the foundation for decision-making and operational tasks, ensuring that the software performs actions that mirror the intended business behavior. This could involve calculating prices, validating transactions, or managing inventory, all based on predefined rules and conditions derived from the organization's requirements.
\hspace{0.5cm}Business logic serves as the intellectual layer of an application, translating business needs into functional processes that can be executed by the software. It defines the constraints, relationships, and actions that underpin the flow of data within the system, ensuring that each operation adheres to the intended policies and delivers accurate results. The clarity and accuracy of business logic are essential for maintaining the reliability of software systems, as it directly influences how well the software aligns with the real-world scenarios it is designed to address. By formalizing business rules into structured logic, it enables organizations to automate and scale their operations effectively while minimizing the risk of errors and inconsistencies.

\hspace{0.5cm} \textbf{Existing method.} The extraction of business logic from source code has been a long-standing challenge, especially in the context of legacy systems. Traditionally, reverse engineering techniques have been employed to bridge the gap between low-level implementation details and high-level conceptual models of software systems. Tools such as SOFT-REDOC have been developed to support this process, particularly for legacy COBOL programs~\cite{BusinessRules}. These tools rely on program stripping, wherein non-essential code is eliminated to focus on the logic that directly affects specific business outcomes. This involves identifying critical variables and their assignments, conditions, and dependencies to reconstruct the underlying business rules.

\hspace{0.5cm} \textbf{Challenges with Existing Approaches.} The reliance on human analysts to interpret outputs and dependencies makes the process time-consuming and error-prone~\cite{BusinessRules}. Furthermore, legacy programs often involve convoluted logic and scattered assignments, making it difficult to reconstruct business rules with precision. In cases where variable names and data structures lack descriptive clarity, analysts may struggle to comprehend the program's intent, leading to incomplete or inaccurate extraction of business logic. These limitations highlight the need for more automated and scalable approaches to understanding business logic in modern and legacy systems.

%% TODO: Make sure to use \subsection{} and \subsubsection{} for smaller sections inside a larger sections
% \section{Theoretical Background}
% %% TODO: Use ~\ref{} to mention a labeled figure, table, section, or anything else
% \subsection{Concept 1}
% Sed eget lobortis leo. Maecenas ut tempor nibh. Nullam arcu nulla, aliquet vel enim maximus, gravida porta tortor. Orci varius natoque penatibus et magnis dis parturient montes, nascetur ridiculus mus. Donec aliquet luctus porttitor. Vivamus porta nulla ut tortor condimentum, at ullamcorper orci facilisis. Vivamus venenatis tellus vel dolor vestibulum, ac placerat orci laoreet. Praesent at elit arcu. Maecenas et lacus sit amet odio finibus semper. Mauris commodo vestibulum aliquam. Mauris non faucibus augue. Vivamus eleifend mauris eget mi venenatis, a maximus dolor mollis in Fig.~\ref{fig:1}.

% %% TODO: Adjust the size of the figure by using [widht=.x\linewidth] (x is a fraction) to fit within the page width. Then rename to your picture file name, add the caption and a label
% \begin{figure}[ht]
% 	\centering
% 	\includegraphics[width=\linewidth]{sample.png}
% 	\caption{The caption of the figure.}
% 	\label{fig:1}
% \end{figure}

% \subsubsection{More details of Concept 1}
% Quisque sit amet ipsum sed ligula congue mattis viverra sit amet sem. Phasellus ante tortor, dictum id ex eget, lacinia pulvinar ligula. Aenean sodales in augue in tempus. Ut ut venenatis magna, feugiat tristique justo. Etiam ac mauris cursus, tincidunt elit commodo, molestie dolor. Nam maximus feugiat nunc, et facilisis eros malesuada vel. Suspendisse potenti. Cras ipsum eros, cursus vitae luctus ac, blandit pulvinar velit. Donec cursus viverra aliquet. Maecenas pharetra nec sem a gravida provided in Table~\ref{tab:1}.

% \begin{table}[ht]
% 	\centering
% 	\caption{Comparison of different methods (\protect\cmark: YES, \protect\xmark: NO).}
% 	\label{tab:1}
% 	%% Comment the next line if the table width is relatively small
% 	\resizebox{\textwidth}{!}{%
% 		\begin{tabular}{lcccccc}
% 			\hline
% 			          & \textbf{Your Method} & Method B & Method C & Method D & Method E & Method F \\ \hline
% 			Feature 1 & \cmark               & \cmark   & \xmark   & \cmark   & \xmark   & \cmark   \\ 
% 			Feature 2 & \cmark               & \xmark   & \cmark   & \cmark   & \cmark   & \xmark   \\ 
% 			Feature 3 & \xmark               & \cmark   & \cmark   & \xmark   & \xmark   & \cmark   \\ 
% 		\end{tabular}%
% 		%%TODO: Also comment this } to match the above command
% 	}
% \end{table}
	
	
% %% TODO: For math mode, wrap the equation inside $ $ for inline mode, wrap inside $$ $$ for non-numbering block mode, or align environment as a numbered block equation
% Ut consectetur quam in elit ullamcorper, non dictum velit congue. Nulla facilisi. Suspendisse potenti. Donec ut felis nec odio tempor rhoncus non a ex $\mathbb{G}_1$ where 

% $$a \in \mathbb{G}_1$$

% Aliquam efficitur fermentum metus, eu posuere orci commodo sit amet. Nullam vulputate consectetur sagittis. Donec imperdiet mi a facilisis facilisis. Cras at diam ornare, suscipit ipsum at, porta arcu. Orci varius natoque penatibus et magnis dis parturient montes, nascetur ridiculus mus. Mauris eu augue quis leo venenatis ultricies. Sed dapibus magna quam, ornare feugiat augue ullamcorper et.

% \begin{align}
% 	e : \mathbb{G}_1 \times \mathbb{G}_2 & \rightarrow \mathbb{G}_T \\
% 	(a, b)                               & \mapsto e(a, b)          
% \end{align}